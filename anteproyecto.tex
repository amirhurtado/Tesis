% --- PREÁMBULO ---
% Aquí se definen el tipo de documento, los paquetes que usaremos, etc.
\documentclass[12pt, letterpaper]{article} % Tipo de documento

% --- PAQUETES ---
\usepackage[utf8]{inputenc} % Permite usar tildes y caracteres en español
\usepackage[spanish]{babel} % Configura el idioma español (para títulos, etc.)
\usepackage{geometry} % Para configurar márgenes
\usepackage{graphicx} % Para incluir imágenes

% Configuración de márgenes
\geometry{left=2.5cm, right=2.5cm, top=3cm, bottom=3cm}

% --- INFORMACIÓN DEL DOCUMENTO ---
\title{Extensión para Visual Studio Code como simulador para procesador RISC-V}
\author{Amir Evelio Hurtado Mena \\ \texttt{a.hurtado@utp.edu.co}}
\date{\today} % Pone la fecha de hoy, puedes cambiarla si quieres


% --- COMIENZO DEL DOCUMENTO ---
\begin{document}

\maketitle % Portada con la información de arriba

\newpage % Crea una nueva página para empezar el contenido

\tableofcontents % Crea el índice

\newpage 

% --- INTRODUCCIÓN ---
\section{Introducción}

La Arquitectura de Computadores es una materia fundamental en la carrera de Ingeniería de Sistemas y Computación, ya que permite entender cómo funciona el hardware que ejecuta todo el software que creamos. Sin embargo, los conceptos sobre el funcionamiento interno de un procesador, como los registros, las unidades de control y los ciclos de instrucción, suelen ser muy abstractos y dificiles de asimilar para los estudiantes que cursan la asignatura en la Universidad Tecnológica de Pereira.

Esta dificultad a menudo genera un obstáculo en el proceso de aprendizaje, ralentizando el avance de las clases y dejando vacíos conceptuales en los futuros profesionales. Para abordar este problema, este proyecto propone el desarrollo de una herramienta de software: una extensión para el editor de código Visual Studio Code que funciona como un simulador gráfico e interactivo de un procesador con arquitectura RISC-V, tanto en su versión monociclo como segmentada.

El objetivo es ofrecer a estudiantes y docentes un recurso que traduzca las operaciones complejas del procesador en visualizaciones claras y fáciles de seguir. De esta manera, se busca fortalecer la comprensión de los temas clave de la materia, haciendo el aprendizaje más práctico, intuitivo y efectivo, y mejorando así la calidad de la formación de los ingenieros de la universidad.
\end{document}